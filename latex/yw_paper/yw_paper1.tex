\message{ !name(yw_paper1.tex)}
%\documentclass[korean,hangul,a4paper, 11pt]{article}
%\documentclass[a4paper, 11pt]{article}
\documentclass[a4paper,twocolumn, 10pt]{article}

%%% 그림 관련
\usepackage[dvips]{graphicx}
\usepackage{subfigure}

% \usepackage{ucs}
% \usepackage[utf8]{inputenc}

% 한국어 문서
% p30
%\usepackage[hangul, nonfrench]{dhucs}

\author{Suil Son, Young-Woon Cha, and Suk I. Yoo}
\title{Abnormal Object Detection Using Second-Order Spatio-Temporal Background Model}

\begin{document}

\message{ !name(yw_paper1.tex) !offset(406) }

The foreground likelihood descriptor \begin{math} h_L(x) \end{math} and the expected likelihood descriptor
\begin{math} h_b(\hat{x}) \end{math} can be built from \begin{math} L(x), E(L(\hat{x})) \end{math} respectively.
The similarity of \begin{math} h_L(x) \end{math}  and \begin{math} h_b(\hat{x}) \end{math} can be obtained using 
\begin{math} \chi ^2 \end{math} test ( 7 ) and is called the second-order foreground likelihood \begin{math} L^2(x) \end{math}.
Also, the positional displacement within one pixel neighbors is considered using ( 12 ).

\ref{fig90} shows an example of variation subtraction process. (a), (b) represent \begin{math} E(\hat{L}) \end{math}.
(d), (g) indicates \begin{math} L \end{math}. (e), (h) are \begin{math} L^2 \end{math}.
By matching histograms between \begin{math} E(\hat{L}) \end{math} and \begin{math} L, L^2 \end{math} can be obtained.
Note that normal dynamic backgrounds are successfully suppressed close to 0, while keeping foregrounds' likelihood from being smoothed.

\message{ !name(yw_paper1.tex) !offset(678) }

\end{document}


