% projectmain = mythesis.tex

\chapter{서론}\label{chap:intro}

\section{연구 목적}

이 논문은 복합문의 통사와 의미의 특성을 밝히는 것을 목적으로 한다. 이러한
일반적인 목적을 앞세우는 이유는 기존의 복합문에 관한 연구에 기초하되 논의를
확장하여 복합문 연구의 새로운 방향을 제시하려는 의도를 가지고 있기 때문이다.
기존의 복합문에 관한 이론적인 기초 연구들이 근본적인 결함을 가지고 있는 것은
아니지만 이중절 복합문만을 관찰 대상으로 삼음으로써 복합문이 가지는 특성의
일면만을 논의하는 한계가 있었다. 이 논문은 다중절 복합문을 본격적으로 분석하여
그 특성을 밝히려는 목적을 가지고 있으며 그 과정에서 삼중절 이상의 복합문에 대한
연구는 이중절 복합문을 중심으로 하는 논의와는 다른 관점의 방향으로 발전될 수
있음이 드러날 것이다.


구체적으로는 복합문의 절 단위 계층 구조 속에서 절 사이의 위계 관계를 분석하는
것이 목적이다. 특히 세 개 이상의 절로 이루어진 복잡한 복합문에서 절 수준의 통사
구조를 분석하는 것이 목적이다. 복합문에 관한 선행 연구들에서는 주로 복합문을
구성하는 하위절의 내부 성분들이 보이는 통사적인 제약 현상을 논의하고 있으며 절
사이의 위계 관계는 주요하게 다루어지지 않았다.


실제 분석에서 제일의 초점은 다중 접속절 복합문의 절 단위 계층 구조를 제약하는
접속어미의 문법적 특성을 밝히는 데에 놓여 있다. 문제를 가장 단순한 형태로
환원하면, 이중 접속절 복합문에서 각 접속절을 이끄는 접속어미만 보면 해당
복합문의 통사 구조를 결정할 수 있도록 하는 언어 정보를 계량적인 근거에 기초하여
구성하는 것이 분석의 목표이다.

이 논문은 한국어에서 빈번하고 자연스럽게 사용되는 다중절 복합문이 어떻게 쉽게
발화되고 이해될 수 있는가라는 질문에 대한 대답을 찾으려는 목적에서 시작되었다.
\citet{YuHyeonJo2004}\와 \citet{YuHyeonJo2007}\은 한국어 대화 전사 자료의 통사
분석을 통해 이 문제에 대한 기본적인 관점을 제시하고 있다.  
구어 발화의 통사 분석을 통해 얻어진 기본적 관점은 복합문을 구성하는 각 절의
명제적 의미를 완전히 해석하지 않고도 그 구조를 판단할 수 있다는 직관적 관찰에
기초하고 있다. 

구어에서는 매우 다양한 어미가 문어에 비해 높은 빈도로 사용되며 통사 분석에서
어미의 역할이 두드러지게 관찰된다. 또한 한국어 문법에서 절의 문법적 특성은
어미의 문법적 특성으로 파악된다. 이 사실에서 복잡한 복합문의  이해와 발화를
가능하게 하는 것은 접속어미로 대표되는 문법 요소에 담겨있는 복합문의 구조에
관한 정보라고 판단하였다.


이 논문의 실제 분석은 세 가지 실질적인 목표로 나뉘어 수행될 것이다.
\ref{chap:patterns}의 문형 분석, \ref{chap:syntax}의 통사 분석,
\ref{chap:semantics}의 의미 분석이 그것이다. 최소한의 언어 형태만을 고려하는
통계적 분석에서 시작하여 점차 더 상세한 문법적 정보를 고려하는 분석으로
나아가는 방향으로 논의가 전개될 것이다.


문형 분석에서는 접속어미가 함께 쓰이는 방식에 일정한 선형 패턴이 있는지
통계적인 방법을 통해 조사할 것이다. 이는 구조를 고려하지 않고 언어 요소들의
선형적인 관계에만 의존하여 복합문을 파악하려는 시도이다. 예를 들어 다음과
같은 접속어미의 연쇄를 고정된 패턴으로 추출할 수 있는지 조사할 것이다.

\ex.  -아서 -으니까 -는다 %



통사 분석에서는 삼중절 복합문에서 통사 구조와 접속어미 사이의 연관성을 조사하여
기초적인 틀을 제시한 후 이로부터 얻어진 분석 방법을 다중절 복합문에 적용하여 절
단위 계층 구조에 대한 제약을 분석할 것이다. 이는 문형 분석과는 달리 구조를
고려하는 분석이다. 예를 들어 다음과 같은 용례의 통사 구조가 두 접속어미
`-아서'와 `-으니까'의 관계에 의해 제약되는 현상을 계량적 분석에 근거하여 논의할
것이다.

\ex. %
\a. [[앉아서] 이야기 하니까] 아무래도 잘 안 돼요.
\b. 어머니 오셨으니까 [[어서 와서] 옷을 갈아입으세요!]


의미 분석에서는 접속어미의 다의성에 따른 복합문 통사 구조의 차이를 분석할
것이다. 이는 형태만을 관찰하는 통사 분석과는 달리 동일한 형태의 어미라도 그
의미에 따라 구별하여 분석하는 것이다. 예를 들어 `-아서'가 계기 관계로 기능하는
경우와 인과 관계로 기능하는 다음 두 용례의 통사 구조의 차이를 분석하여 접속의
의미가 복합문의 계층 구조를 제약하는 현상을 논의할 것이다.

\ex.
\a. 배가 고프니까  [[저 혼자 나가서] 먹이를 구해 왔잖니.] (계기의 `-아서')
\b. 나는 [[양주를 보니까] 생각이 나서] 이런 말을 꺼냈다. (인과의 `-아서')


\section{연구 대상}

이 논문의 연구 대상은 언어 범주의 측면에서 넓게는 복합문이며 좁게는
접속어미이다. 언어 현상의 측면에서는 절 단위 계층 구조의 제약 현상을 제일 연구
대상으로 한정하되 통사와 의미의 상호작용을 시야에 둘 것이다. 개별 언어의
측면에서 연구 대상은 현대 문어 한국어이며, 실제 언어 자료로는 21세기 세종
계획의 현대국어 기초 말뭉치로 구축된 형태의미분석 말뭉치와 구문분석 말뭉치를
이용할 것이다.


복합문이 한국어에서 매우 높은 빈도로 자연스럽게 사용되는 것은 한국어의 고유한
특성 중의 하나이다. 이러한 특성 때문에 알타이언어들을 비롯한 몇몇 언어들과 함께
한국어를 부동사(converb)를 가지는 언어로 분류하는 논의가 이루어지기도 하였다.
복합문의 잦은 사용과 이를 구성하는 어미들의 다양한 형태에서 예상할 수 있듯이
국어 전통 문법의 성립 이래로 복합문은 한국어 문법의 주요 범주의 하나로 논의되어
왔다.

한국어에서 복합문의 통사와 의미는 단순문의 통사와 의미에 대한 연구에 못지 않은
핵심 주제 중의 하나에 속한다. 한국어의 다양한 접속어미를 고려하고 서너 개의
절이 연결되는 문장이 일반적이고 자연스럽다는 점을 고려하면 한국어에서 복합문
연구는 통사론와 의미론의 핵심적이고 본질적인 문제에 매우 가까이 있다고 판단할
수 있다. 복합문의 하위절에서 주어와 논항이 명시적으로 실현되지 않는 경우가
흔하다는 것을 함께 고려하면 명사구 없이 여러 개의 술어들이 접속으로 길게 나열된
문장도 쉽게 상정할 수 있다. 실제로 자연스러운 한국어 발화에서는 술어, 논항,
부가어 사이의 관계보다 절과 절의 관계가 많은 경우를 다음과 같은 예문에서 쉽게
찾아볼 수 있다.


\ex. 지금 회의가 있어서 나가 봐야 되니까 마무리 되는 대로 출력해서
책상 위에 두고 가시면 내일 와서 읽어 보고 전화 드릴게요.


이 논문에서는 이러한 예문으로 대표될 수 있는 다중절 복합문을 대상으로 절 단위
계층 구조의 제약 현상을 분석할 것이다. 위 예문의 통사 구조는 9개의 절이
일차원적으로 연결되어 있는 구조가 아니라 2차원적인 계층 구조로 파악된다.
통사론과 의미론의 차원에서 복잡한 복합문의 계층 구조를 결정하는 언어학적인
특징들을 분석할 것이다. 어떤 절들이 먼저 묶이고 어떤 절들이 나중에 묶이는지
한국어 화자는 듣는 것과 동시에 파악을 한다. 인지적인 관점에서는 한국어 모국어
화자가 어떻게 그 구조를 쉽게 알아내는지 연구할 수 있다. 구문분석의 관점에서는
이런 현상을 형식적인 언어로 기술하거나 통계적인 정보를 제공하여 자동으로 분석할
수 있는 틀을 만드는 연구를 할 수 있다. 이 논문은 언어학적인 관점을 가지고
이러한 문장에서 한국어 화자가 구조를 찾아낼 수 있는 것은 거기에 어떠한 언어적인
정보가 있기 때문이라고 가정한다. 접속절의 문법적 특성은 접속어미의 문법적
특성으로 파악된다. 이러한 관점에서 절 단위 계층 구조는 접속어미의 문법적 특성에
의해 제약될 것이라는 가설을 세우고 이를 검증할 것이다.

이 논문의 논의는 절 수준 통사론과 의미론이며, 절의 내부를 관찰하지 않고 절을
하나의 덩어리로 보아 절과 절 사이의 관계를 분석할 것이다. 복합문의 통사와
의미를 연구하는 것은 복합문이라는 구조에 나타나는 고유한 현상을 다루는 것이다.
절과 절 사이의 관계가 바로 그 대상이다. 각 절의 내부 구조는 단순문의 연구에서
다루어진다. 복합문이라는 언어 범주의 특성은 여러 개의 절로 이루어졌다는
것이다. 그것을 구성하는 절들이 맺고 있는 관계가 바로 복합문의 통사와 의미에
관한 논의 대상이다.

복합문의 구조적 제약 현상의 분석에는 절을 하나의 덩어리로 보는 분석 외에 한
절의 구성 성분이 다른 절 또는 다른 절의 구성 성분과 상호작용하는 현상을
분석하는 것이 있다. 이는 절의 구성 성분이 절의 경계를 넘어 관계를 맺는 현상을
분석하는 것이다. 복합문 관련 선행 연구에서는 주어의 인칭 제약, 시제법 제약,
주절의 의향법 제약 등 이에 해당하는 기초 분석들이 주로 다루어졌다. 이
논문에서는 절을 구성하는 성분의 제약 현상은 분석하지 않을 것이다.



\section{연구 방법}

이 논문은 말뭉치 기반 연구이다. 실제 분석에서 말뭉치를 적극적으로
광범위하게 이용할 것이다. 이 논문에 서술된 분석 이전에 연구의 방향을 마련하기
위해 수행한 기초 작업은 말뭉치 주도 접근법이라고 할 수 있으나, 그러한 부분은 본론에서
논의하지 않을 것이다. 

% Elena Tognini-Bonelli (2001), Corpus Linguistics at Work, Amsterda: John Benjamins
\citet{Tognini-Bonelli2001}\는 말뭉치를 이용한 연구를 크게 말뭉치 기반
접근법(corpus based approach)과 말뭉치 주도 접근법(corpus driven approach)으로
구분하고 있다. 말뭉치 기반 방법론은 말뭉치에서 적절한 자료를 뽑아내어 직관적
언어 지식의 근거로 삼고, 예측을 검증하고, 언어 현상을 수량화하고, 이론을
검증하는 도구로 삼는 방법론을 말한다. 말뭉치 주도 접근법은 어떠한 선행 가정이나
예측 없이 말뭉치에서 자료를 뽑아내어 언어학적 현상을 포착하는 방법론을 말한다.


이 논문의 연구는 말뭉치에서 용례를 찾아 근거로 삼는 수준을 넘어 다량의 자료를
추출하여 계량적 분석을 수행하고 있어 논의가 말뭉치에 크게 의존하고 있다. 말뭉치
기반 연구는 논거가 되는 자료를 직관에 의존해서 만들어서 사용하는 것이 아니라
말뭉치에서 논거를 찾는다. 말뭉치 분석이 연구에서 어느 정도의 비중을
차지하느냐는 연구에 따라 상당한 차이가 있을 수 있다.  이 논문의 4장과 5장은
말뭉치에서 뽑아낸 자료의 계량적 분석에 근거하여 논의를 전개하고 있어 말뭉치
주도 접근법에 근접할 만큼 말뭉치가 크게 비중을 차지는 편에 속한다.

모어 화자로서 필자의 직관을 이용해 비문법의 문턱에 놓인 예문을 만들어 실험하는
방법론은 취하지 않을 것이다. 말뭉치 기반 연구는 논거가 되는 자료를 직관에
의존해서 만들어서 사용하는 것이 아니라 말뭉치에서 논거를 찾는다는 점에서 차이가
있다. 이 논문에도 선행 연구를 인용하는 경우 등에서 작례를 이용해 논의를 전개한
부분이 일부 있으나 그 문법성 자체가 논의거리가 되는 경우는 없을 것이다.

말뭉치 기반 연구가 직관을 이용한 논의와 서로 대립되는 것으로 인식되는 경우가
있으나 말뭉치 기반 연구가 직관을 부정하는 것은 아니다. 말뭉치라는 것이
언어능력을 가진 화자에 의해 만들어진 발화체들의 모음이므로 당연히 그 발화체들은
화자의 직관에 의해서 만들어진 것이다. 말뭉치를 이용하는 연구자 또한 자신의
직관에 기초해서 자료를 분석하기 위한 틀을 만든다. 말뭉치 기반 연구가 언어
연구에서 직관을 중요시하는 관점에 배치되는 것은 아니다. 

말뭉치 기반 접근법은 연구자가 의도적으로 혹은 무의식적으로 자신의 직관을 왜곡해서
자신의 관점에 맞는 발화체를 만들어 내는 오류를 피하고 분석을 통한 객관적 근거를
제시할 수 있게 해 준다는 점에서 그 효용성이 있다.




%%% Local Variables: 
%%% mode: latex
%%% TeX-master: "mythesis"
%%% End: 

