% -*- mode : latex; coding: utf-8 -*- projectmain = mythesis.tex
%%
%% Thesis 2007
%% written by You, Hyun Jo
%%
%% @PhDThesis{
%%		author  = {유현조},
%%		title   = {한국어 복합문의 통사와 의미},
%%		school  = {서울대학교},
%%		year    = {2008},
%%		month   = {August},
%%		type    = {박사학위논문},
%%		keyword = {한국어,복합문,어미,다중절문,어미의 통사적 제약},
%%	}
%%

%% 최종 출력
\documentclass[final,korean,hangul,nonfrench,finemath,a4paper,11pt,twoside,openright]{snuthesis}

%% 화면으로 보기에 편리한 PDF
%\documentclass[screen,korean,hangul,11pt,twoside,openright]{snuthesis}

\usepackage{mythesis}
\usepackage{fancyhdr}
\usepackage{datetime} % for \pdfdate

%%
%% Declaration
%%
    %
    \author{유현조}
    %\othername{You, Hyun Jo}
    \studentid{2003-30026}

    \title{한국어 복합문의 통사와 의미}
    \subtitle{--- 절 단위 계층 구조에 관한 말뭉치 기반 연구 ---}
    %\othertitle{Syntax and Semantics of Korean Complex Sentence: \\
    %         A Study of Structural Constraints on Verbal Endings}

    \degree{문학박사}
	\degreedate{2008년\hspace{1pc}8월}
    \degreeyear{2008}
    \degreemonth{8}

    \department{언어학과}
    \major{언어학}


    \submitdate{2008}{4}
    \signdate{2008}{7}

    \advisor{권재일}

    \supervisorA{남승호}
    \supervisorB{권재일}
    \supervisorC{김주원}
    \supervisorD{전영철}
    \supervisorE{목정수}



%%================================================================
%% begin document
%%================================================================
%\date{2008년 7월 7일}
\begin{document}



%\layout
\maketitle%표지와 인준지 출력
\makesignedpage


\newpage
\cleardoublepage
% pagestyle: header and footer
\pagestyle{fancy}
\fancyhead[RO,LE]{\thepage}
\fancyhead[LO]{\rightmark}
\fancyhead[RE]{\leftmark}
\fancyfoot{}
\newcommand{\cedp}{%
	\newpage{\pagestyle{empty}\cleardoublepage}
}

%국문초록
%% projectmain = mythesis.tex
%%% 국문초록
\begin{summary}
%
% 배경: 왜 시작했는가?
%

\par

한국어 복합문은 문어 텍스트에서 문장의 대다수를 차지하며 구어 발화에서도
상대적으로 비중은 낮으나 여전히 높은 빈도로 자연스럽게 사용된다. 이러한 복합문의
사용은 한국어의 특징적인 현상이다. 이론적인 차원에서 복합문의 통사와 의미는
국문법의 중요한 주제의 하나로 연구되어 왔으며 실제적인 차원에서 복합문의 구문
분석은 가장 어려운 문제의 하나로 인식되고 있다.

%
% 문제 제기
%

이러한 중요성에도 불구하고 다중절 복합문의 구조적 특성과 중의성 해소에
대해서는 깊은 언어학적 논의가 이루어지지 않았다. 기존의 연구들은 가장 단순한
구성인 이중절 복합문을 논의 대상으로 하며 문장 성분들 사이의 통사적 제약에
분석의 초점을 두고 있다. 세 개 이상의 절로 이루어진 복합문의 구조는 주요하게
다루어지지 않았다.


%
% 목적: 무엇을 하려고 했는가?
%

이 논문은 다중절 복합문을 대상으로 절과 절 사이의 관계를 분석하여 절 단위 계층
구조를 분석하는 것을 목표로 하였다. 이 논문은 다중 접속절 복합문의 구조를
이해하는 데에 관여하는 언어적 정보가 무엇인지 밝히는 것에 관심을 가지고 있다.
한국어 문법에서 한 절의 문법적 특성은 용언에 결합한 어미의 특성으로 파악되므로
어미가 복합문의 구조를 제약하는 현상을 분석하는 것이 제일의 목표가 되었다.
어미를 절을 이끄는 가장 대표적인 요소로 주요하게 다루되 접속어미뿐만 아니라
접속의 기능을 하는 `-을 때', `-는 동안', `-기 때문에' 등의 접속기능구성도 분석
대상으로 삼았다.

%
% 방법: 무엇을 했는가?
%

실제 분석에서는 21세기 세종계획의 현대국어 기초 말뭉치를 이용하였으며 다중
접속절 복합문의 통사 구조가 접속어미에 의해 제약되는 현상에 초점을 두고
문형 분석, 통사 분석, 의미 분석을 수행하였다. 이러한 분석은 언어 형태만을
고려하는 통계적인 분석에서 점차 상세한 문법 정보를 이용한 정밀한 관찰로
나아가며 수행된 것이다. 4장의 문형 분석에서는 복합문의 문형을 각 하위절을
이끄는 어미들의 연쇄 패턴으로 파악하여 분석하였고, 5장의 통사 분석에서는 삼중절
복합문의 통사 구조와 두 접속어미 사이의 위계 관계 분석을 기초로 하여 다중절
복합문에서 절 사이의 위계 관계를 분석하였으며, 6장의 의미 분석에서는 접속어미의
다의성에 따른 통사 구조의 제약을 분석하였다.

%
% 결과: 무엇을 발견했는가?
%

문형 분석 결과는 다중절 복합문에 어미들의 고정된 연쇄 패턴이 존재하지
않으며, 접속어미들의 선형 결합 양상이 매우 자유롭다는 것을 보여주었다.


통사 분석 결과는 접속어미 사이에 일정한 위계 관계가 존재하며 다중 접속절
복합문의 절 단위 계층 구조가 각 접속절을 이끄는 접속어미 사이의 위계 관계에
따라 결정됨을 보여주었다. 예를 들어 세 개의 절로 이루어진 문장 `S$_1$-$c_1$
S$_2$-$c_2$ S$_3$'의 구조는 접속어미 $c_1$와 $c_2$의 위계 관계에 따라
제약된다.

\ex.
\a. $c_1 < c_2$ $\rightarrow$ [[S$_1$-$c_1$ S$_2$-$c_2$] S$_3$]
\b. $c_1 > c_2$ $\rightarrow$ [S$_1$-$c_1$ [S$_2$-$c_2$ S$_3$]]


의미 분석 결과는 이러한 복합문의 구조적 제약 현상이 접속어미라는 문법 형태
사이의 위계 관계가 아니라 접속의 의미 사이의 위계 관계에 의한 것임을
보여주었다. 예를 들어 다의적 접속어미의 하나인 `-아서'가 인과 관계로 기능할
때와 계기 관계로 기능할 때 보이는 통사 구조의 차이가 이를 분명히 드러낸다.
 
이러한 분석 결과가 의미하는 것은 접속의 의미가 이루는 위계 구조가 절 단위 계층
구조를 제약한다는 것, 즉 절 수준의 의미가 복합문의 통사 구조를 제약한다는 것이다.
이 논문에서 관찰한 언어 자료의 테두리 내에서 접속의 의미들
사이에 다음과 같은 위계가 성립하는 것을 확인하였다.

\ex. 대조 > 양보 > 조건 > 중단 > 동시 > 계기

접속의 의미 사이의 위계는 담화 구조의 논리적 분석에서 사용하는 보편적인 개념을
빌어 일반화될 수 있다. 대조는 닮음 관계, 양보와 조건은 인과 관계, 중단, 동시,
계기는 인접 관계에 해당한다.


\ex. 닮음 관계 > 인과 관계 > 인접 관계


%
% 결론: 그게 무슨 의미인가?
%


복합문 내의 접속이 담화 내의 연결과 동일한 성격을 가진다는 사실은 다중절
복합문과 다중문 담화를 동일한 원리로 분석하는 방법론의 타당성을 뒷받침한다.
이를 바탕으로 이 논문에서는 접속어미를 함수로 보고 이에 의해 연결되는 두
절을 논항으로 보는 D-LTAG, SDRT, RST 등의 관점에 동의하며 한국어 복합문의 통사
구조와 의미 구조를 분석하고 형식화하기 위한 틀을 제안하였다.

\vfill

\bigskip
\noindent
주요어: 한국어, 복합문, 절 수준 통사, 담화 의미론, 말뭉치, 접속어미\\
학\hspace{11pt}번: 2003--30026

\end{summary}



%%% Local Variables: 
%%% mode: latex
%%% TeX-master: "mythesis"
%%% End: 

\cedp




\makecontents
\cedp
%\tableofcontents
%\listoffigures
%\listoftables

%\include{test}

%\citeindextrue


%================================================================
% End-Of-Header
%================================================================



% projectmain = mythesis.tex

\chapter{서론}\label{chap:intro}

\section{연구 목적}

이 논문은 복합문의 통사와 의미의 특성을 밝히는 것을 목적으로 한다. 이러한
일반적인 목적을 앞세우는 이유는 기존의 복합문에 관한 연구에 기초하되 논의를
확장하여 복합문 연구의 새로운 방향을 제시하려는 의도를 가지고 있기 때문이다.
기존의 복합문에 관한 이론적인 기초 연구들이 근본적인 결함을 가지고 있는 것은
아니지만 이중절 복합문만을 관찰 대상으로 삼음으로써 복합문이 가지는 특성의
일면만을 논의하는 한계가 있었다. 이 논문은 다중절 복합문을 본격적으로 분석하여
그 특성을 밝히려는 목적을 가지고 있으며 그 과정에서 삼중절 이상의 복합문에 대한
연구는 이중절 복합문을 중심으로 하는 논의와는 다른 관점의 방향으로 발전될 수
있음이 드러날 것이다.


구체적으로는 복합문의 절 단위 계층 구조 속에서 절 사이의 위계 관계를 분석하는
것이 목적이다. 특히 세 개 이상의 절로 이루어진 복잡한 복합문에서 절 수준의 통사
구조를 분석하는 것이 목적이다. 복합문에 관한 선행 연구들에서는 주로 복합문을
구성하는 하위절의 내부 성분들이 보이는 통사적인 제약 현상을 논의하고 있으며 절
사이의 위계 관계는 주요하게 다루어지지 않았다.


실제 분석에서 제일의 초점은 다중 접속절 복합문의 절 단위 계층 구조를 제약하는
접속어미의 문법적 특성을 밝히는 데에 놓여 있다. 문제를 가장 단순한 형태로
환원하면, 이중 접속절 복합문에서 각 접속절을 이끄는 접속어미만 보면 해당
복합문의 통사 구조를 결정할 수 있도록 하는 언어 정보를 계량적인 근거에 기초하여
구성하는 것이 분석의 목표이다.

이 논문은 한국어에서 빈번하고 자연스럽게 사용되는 다중절 복합문이 어떻게 쉽게
발화되고 이해될 수 있는가라는 질문에 대한 대답을 찾으려는 목적에서 시작되었다.
\citet{YuHyeonJo2004}\와 \citet{YuHyeonJo2007}\은 한국어 대화 전사 자료의 통사
분석을 통해 이 문제에 대한 기본적인 관점을 제시하고 있다.  
구어 발화의 통사 분석을 통해 얻어진 기본적 관점은 복합문을 구성하는 각 절의
명제적 의미를 완전히 해석하지 않고도 그 구조를 판단할 수 있다는 직관적 관찰에
기초하고 있다. 

구어에서는 매우 다양한 어미가 문어에 비해 높은 빈도로 사용되며 통사 분석에서
어미의 역할이 두드러지게 관찰된다. 또한 한국어 문법에서 절의 문법적 특성은
어미의 문법적 특성으로 파악된다. 이 사실에서 복잡한 복합문의  이해와 발화를
가능하게 하는 것은 접속어미로 대표되는 문법 요소에 담겨있는 복합문의 구조에
관한 정보라고 판단하였다.


이 논문의 실제 분석은 세 가지 실질적인 목표로 나뉘어 수행될 것이다.
\ref{chap:patterns}의 문형 분석, \ref{chap:syntax}의 통사 분석,
\ref{chap:semantics}의 의미 분석이 그것이다. 최소한의 언어 형태만을 고려하는
통계적 분석에서 시작하여 점차 더 상세한 문법적 정보를 고려하는 분석으로
나아가는 방향으로 논의가 전개될 것이다.


문형 분석에서는 접속어미가 함께 쓰이는 방식에 일정한 선형 패턴이 있는지
통계적인 방법을 통해 조사할 것이다. 이는 구조를 고려하지 않고 언어 요소들의
선형적인 관계에만 의존하여 복합문을 파악하려는 시도이다. 예를 들어 다음과
같은 접속어미의 연쇄를 고정된 패턴으로 추출할 수 있는지 조사할 것이다.

\ex.  -아서 -으니까 -는다 %



통사 분석에서는 삼중절 복합문에서 통사 구조와 접속어미 사이의 연관성을 조사하여
기초적인 틀을 제시한 후 이로부터 얻어진 분석 방법을 다중절 복합문에 적용하여 절
단위 계층 구조에 대한 제약을 분석할 것이다. 이는 문형 분석과는 달리 구조를
고려하는 분석이다. 예를 들어 다음과 같은 용례의 통사 구조가 두 접속어미
`-아서'와 `-으니까'의 관계에 의해 제약되는 현상을 계량적 분석에 근거하여 논의할
것이다.

\ex. %
\a. [[앉아서] 이야기 하니까] 아무래도 잘 안 돼요.
\b. 어머니 오셨으니까 [[어서 와서] 옷을 갈아입으세요!]


의미 분석에서는 접속어미의 다의성에 따른 복합문 통사 구조의 차이를 분석할
것이다. 이는 형태만을 관찰하는 통사 분석과는 달리 동일한 형태의 어미라도 그
의미에 따라 구별하여 분석하는 것이다. 예를 들어 `-아서'가 계기 관계로 기능하는
경우와 인과 관계로 기능하는 다음 두 용례의 통사 구조의 차이를 분석하여 접속의
의미가 복합문의 계층 구조를 제약하는 현상을 논의할 것이다.

\ex.
\a. 배가 고프니까  [[저 혼자 나가서] 먹이를 구해 왔잖니.] (계기의 `-아서')
\b. 나는 [[양주를 보니까] 생각이 나서] 이런 말을 꺼냈다. (인과의 `-아서')


\section{연구 대상}

이 논문의 연구 대상은 언어 범주의 측면에서 넓게는 복합문이며 좁게는
접속어미이다. 언어 현상의 측면에서는 절 단위 계층 구조의 제약 현상을 제일 연구
대상으로 한정하되 통사와 의미의 상호작용을 시야에 둘 것이다. 개별 언어의
측면에서 연구 대상은 현대 문어 한국어이며, 실제 언어 자료로는 21세기 세종
계획의 현대국어 기초 말뭉치로 구축된 형태의미분석 말뭉치와 구문분석 말뭉치를
이용할 것이다.


복합문이 한국어에서 매우 높은 빈도로 자연스럽게 사용되는 것은 한국어의 고유한
특성 중의 하나이다. 이러한 특성 때문에 알타이언어들을 비롯한 몇몇 언어들과 함께
한국어를 부동사(converb)를 가지는 언어로 분류하는 논의가 이루어지기도 하였다.
복합문의 잦은 사용과 이를 구성하는 어미들의 다양한 형태에서 예상할 수 있듯이
국어 전통 문법의 성립 이래로 복합문은 한국어 문법의 주요 범주의 하나로 논의되어
왔다.

한국어에서 복합문의 통사와 의미는 단순문의 통사와 의미에 대한 연구에 못지 않은
핵심 주제 중의 하나에 속한다. 한국어의 다양한 접속어미를 고려하고 서너 개의
절이 연결되는 문장이 일반적이고 자연스럽다는 점을 고려하면 한국어에서 복합문
연구는 통사론와 의미론의 핵심적이고 본질적인 문제에 매우 가까이 있다고 판단할
수 있다. 복합문의 하위절에서 주어와 논항이 명시적으로 실현되지 않는 경우가
흔하다는 것을 함께 고려하면 명사구 없이 여러 개의 술어들이 접속으로 길게 나열된
문장도 쉽게 상정할 수 있다. 실제로 자연스러운 한국어 발화에서는 술어, 논항,
부가어 사이의 관계보다 절과 절의 관계가 많은 경우를 다음과 같은 예문에서 쉽게
찾아볼 수 있다.


\ex. 지금 회의가 있어서 나가 봐야 되니까 마무리 되는 대로 출력해서
책상 위에 두고 가시면 내일 와서 읽어 보고 전화 드릴게요.


이 논문에서는 이러한 예문으로 대표될 수 있는 다중절 복합문을 대상으로 절 단위
계층 구조의 제약 현상을 분석할 것이다. 위 예문의 통사 구조는 9개의 절이
일차원적으로 연결되어 있는 구조가 아니라 2차원적인 계층 구조로 파악된다.
통사론과 의미론의 차원에서 복잡한 복합문의 계층 구조를 결정하는 언어학적인
특징들을 분석할 것이다. 어떤 절들이 먼저 묶이고 어떤 절들이 나중에 묶이는지
한국어 화자는 듣는 것과 동시에 파악을 한다. 인지적인 관점에서는 한국어 모국어
화자가 어떻게 그 구조를 쉽게 알아내는지 연구할 수 있다. 구문분석의 관점에서는
이런 현상을 형식적인 언어로 기술하거나 통계적인 정보를 제공하여 자동으로 분석할
수 있는 틀을 만드는 연구를 할 수 있다. 이 논문은 언어학적인 관점을 가지고
이러한 문장에서 한국어 화자가 구조를 찾아낼 수 있는 것은 거기에 어떠한 언어적인
정보가 있기 때문이라고 가정한다. 접속절의 문법적 특성은 접속어미의 문법적
특성으로 파악된다. 이러한 관점에서 절 단위 계층 구조는 접속어미의 문법적 특성에
의해 제약될 것이라는 가설을 세우고 이를 검증할 것이다.

이 논문의 논의는 절 수준 통사론과 의미론이며, 절의 내부를 관찰하지 않고 절을
하나의 덩어리로 보아 절과 절 사이의 관계를 분석할 것이다. 복합문의 통사와
의미를 연구하는 것은 복합문이라는 구조에 나타나는 고유한 현상을 다루는 것이다.
절과 절 사이의 관계가 바로 그 대상이다. 각 절의 내부 구조는 단순문의 연구에서
다루어진다. 복합문이라는 언어 범주의 특성은 여러 개의 절로 이루어졌다는
것이다. 그것을 구성하는 절들이 맺고 있는 관계가 바로 복합문의 통사와 의미에
관한 논의 대상이다.

복합문의 구조적 제약 현상의 분석에는 절을 하나의 덩어리로 보는 분석 외에 한
절의 구성 성분이 다른 절 또는 다른 절의 구성 성분과 상호작용하는 현상을
분석하는 것이 있다. 이는 절의 구성 성분이 절의 경계를 넘어 관계를 맺는 현상을
분석하는 것이다. 복합문 관련 선행 연구에서는 주어의 인칭 제약, 시제법 제약,
주절의 의향법 제약 등 이에 해당하는 기초 분석들이 주로 다루어졌다. 이
논문에서는 절을 구성하는 성분의 제약 현상은 분석하지 않을 것이다.



\section{연구 방법}

이 논문은 말뭉치 기반 연구이다. 실제 분석에서 말뭉치를 적극적으로
광범위하게 이용할 것이다. 이 논문에 서술된 분석 이전에 연구의 방향을 마련하기
위해 수행한 기초 작업은 말뭉치 주도 접근법이라고 할 수 있으나, 그러한 부분은 본론에서
논의하지 않을 것이다. 

% Elena Tognini-Bonelli (2001), Corpus Linguistics at Work, Amsterda: John Benjamins
\citet{Tognini-Bonelli2001}\는 말뭉치를 이용한 연구를 크게 말뭉치 기반
접근법(corpus based approach)과 말뭉치 주도 접근법(corpus driven approach)으로
구분하고 있다. 말뭉치 기반 방법론은 말뭉치에서 적절한 자료를 뽑아내어 직관적
언어 지식의 근거로 삼고, 예측을 검증하고, 언어 현상을 수량화하고, 이론을
검증하는 도구로 삼는 방법론을 말한다. 말뭉치 주도 접근법은 어떠한 선행 가정이나
예측 없이 말뭉치에서 자료를 뽑아내어 언어학적 현상을 포착하는 방법론을 말한다.


이 논문의 연구는 말뭉치에서 용례를 찾아 근거로 삼는 수준을 넘어 다량의 자료를
추출하여 계량적 분석을 수행하고 있어 논의가 말뭉치에 크게 의존하고 있다. 말뭉치
기반 연구는 논거가 되는 자료를 직관에 의존해서 만들어서 사용하는 것이 아니라
말뭉치에서 논거를 찾는다. 말뭉치 분석이 연구에서 어느 정도의 비중을
차지하느냐는 연구에 따라 상당한 차이가 있을 수 있다.  이 논문의 4장과 5장은
말뭉치에서 뽑아낸 자료의 계량적 분석에 근거하여 논의를 전개하고 있어 말뭉치
주도 접근법에 근접할 만큼 말뭉치가 크게 비중을 차지는 편에 속한다.

모어 화자로서 필자의 직관을 이용해 비문법의 문턱에 놓인 예문을 만들어 실험하는
방법론은 취하지 않을 것이다. 말뭉치 기반 연구는 논거가 되는 자료를 직관에
의존해서 만들어서 사용하는 것이 아니라 말뭉치에서 논거를 찾는다는 점에서 차이가
있다. 이 논문에도 선행 연구를 인용하는 경우 등에서 작례를 이용해 논의를 전개한
부분이 일부 있으나 그 문법성 자체가 논의거리가 되는 경우는 없을 것이다.

말뭉치 기반 연구가 직관을 이용한 논의와 서로 대립되는 것으로 인식되는 경우가
있으나 말뭉치 기반 연구가 직관을 부정하는 것은 아니다. 말뭉치라는 것이
언어능력을 가진 화자에 의해 만들어진 발화체들의 모음이므로 당연히 그 발화체들은
화자의 직관에 의해서 만들어진 것이다. 말뭉치를 이용하는 연구자 또한 자신의
직관에 기초해서 자료를 분석하기 위한 틀을 만든다. 말뭉치 기반 연구가 언어
연구에서 직관을 중요시하는 관점에 배치되는 것은 아니다. 

말뭉치 기반 접근법은 연구자가 의도적으로 혹은 무의식적으로 자신의 직관을 왜곡해서
자신의 관점에 맞는 발화체를 만들어 내는 오류를 피하고 분석을 통한 객관적 근거를
제시할 수 있게 해 준다는 점에서 그 효용성이 있다.




%%% Local Variables: 
%%% mode: latex
%%% TeX-master: "mythesis"
%%% End: 

%서론
\cedp
\include{2._include_}%개념
\cedp
\include{3._include_}%이론
\cedp
\include{4._include_}%문형
\cedp
\include{5._include_}%통사 분석
\cedp
\include{6._include_}%의미 유형
\cedp
\include{7._include_}%통사-의미 구조
\cedp
\include{8._include_}%결론
\cedp


%%================================================================
%% appendix 부록
%%================================================================
\appendix

\include{A._include_}%부록
\cedp



%================================================================
% Start-Of-Trailer
%================================================================


%%================================================================
%% references 참고문헌
%%================================================================
%\bibliographystyle{chicago}%{plainnat}%{apacite}%{fullcite}
%\bibliography{references}
\clearpage
\phantomsection
\addcontentsline{toc}{chapter}{참고 문헌}
\input references.bbl
\cedp


%%================================================================
%% index 찾아보기
%%================================================================
%\clearpage
%\phantomsection
%\addcontentsline{toc}{chapter}{찾아보기}
%\printindex
%\cedp

%%================================================================
%% abstract 영문초록
%%================================================================
\clearpage
\phantomsection
%% projectmain = mythesis.tex
\pagestyle{plain}
\setlength\parindent{1.5pc}% 들여쓰기 1글자
\begin{abstract}


\begin{center}
\normalsize\sc
{\maketitlebreakwork\bfseries
The Syntax and Semantics of Korean Complex Sentences:\\
\small
a corpus based study of the hierarchical structure of interclausal relations\\
\maketitlebreaknotwork
}

\bigskip\small
by\\
\normalsize
YOU Hyun Jo\\
\bigskip
\small
Under the Supervision of Professor KWON Jae-il\\
Submitted to the Department of Linguistics, Seoul National University,\\
in partial fulfillment of the requirements for the degree of\\
Doctor of Philosophy
\bigskip

August 2008
\end{center}
\bigskip

% Movitation
\noindent
Korean complex sentences are exceedingly dominant in written texts and
reportedly not dominant but still  frequent and natural in spoken
utterances. The verbal endings play the key role in the syntax and semantics
of the multi-clausal structures. The clause combining with `connective' verbal
endings or `converb' forms has been studied as one of the most important
syntactic phenomena since the existence of the traditional Korean grammar and
considered as one of the most challenging problems of parsing in recent
computational linguistic research.

% Problem statement
The syntax of multi-clausal sentences has hardly been studied in traditional
and modern grammars, in which the discussion has usually been limited to
describing two-clause sentences and the semantic function of verbal endings. A
complex sentence with one subordinate clause has the evident clause level syntax
but one with multiple subordinate clauses has enormous ambiguity exactly like
in the classical PP attachment problem.

%  What did you try to do? What did yo do?
In this paper I have attempted to uncover what linguistic information is
relevant to dealing with structural ambiguity of the complex sentences with
multiple adjunct `connective' clauses. In Korean grammar, the grammatical
properties of a clause are viewed as properties of the verb and its ending.
The corpus analysis was designed to test the hypothesis that the connectives
determine the clause level syntax.

This study investigated the restriction of the connective verbal endings on
the structure of the multi-clausal sentences in large corpora. I computed the
mutual information between the connective endings of adjacent clauses for
identifying the pattern of clause combining in chapter 4, and extracted
syntactic dependencies between connective clauses from parsed corpora for
constructing the hierarchy of connective endings in chapter 5, and examined
carefully how the polysemy of connective endings affect the syntactic
structure of the sentence using manually annotated sentences in chpter 6.

% What did you find
I found no clause combining patterns but consistent hierarchical relationships
between connective verbal endings. The relative ranks of the endings restrict
the clause combining. The syntactic structure of a sentence with two
subordinate clauses `S$_1$-$c_1$ S$_2$-$c_2$ S$_3$' is restricted by the
relationship between the two connectives $c_1$ and $c_2$: `[[S$_1$-$c_1$
S$_2$-$c_2$] S$_3$]' if $c_1 < c_2$, `[S$_1$-$c_1$ [S$_2$-$c_2$ S$_3$]]' if
$c_1 > c_2$. The hierarchy is determined according to the order of the
semantic functions of the connectives: Contrast > Concession > Condition >
Interruption > Concurrence > Succession. It can be generalized to the
hierarchy of coherence relations in the terms of David Hume: Resemblance >
Cause and Effect > Contiguity.

% What does it means?
It means that we can apply same rule to the analysis of the multi-clausal
sentential structures and the multi-sentential discourse structures. The rule
is a specification of the semantic requirement of the connective
(the predicate) on its argument (the clauses). I proposed some methods for dealing
with syntactic and semantic analysis of Korean complex sentences based on the
philosophy of D-LTAG (Discourse-Lexicalized Tree Adjoining Grammar), SDRT
(Segmented Discourse Representation Theory) and RST (Rhetorical Structure
Theory).

\vfill
\noindent
Keywords: Korean language, complex sentences, 
clause-level syntax, discourse semantics, corpora (linguistics), interclausal
relationships, connectives (linguistics)\\
Student Number: 2003--30026


\end{abstract}

%%% Local Variables: 
%%% mode: latex
%%% TeX-master: "mythesis"
%%% End: 

\cedp



%%================================================================
%% acknowledgement 감사의 글
%%================================================================
%\begin{acknowledgement}
%이것은 감사의 글입니다.
%\end{acknowledgement}




%%================================================================
%% PDF Info
%%================================================================
\newpage
\thispagestyle{empty}

\pdfinfo{
/Author (YOU Hyun Jo)
/Title (The syntax and semantics of Korean complex sentences: a corpus based approach of the hierarchical structure of interclausal relations)
/CreationDate (D:\pdfdate)
/ModificationDate (D:\pdfdate)
/Subject ()
/Keywords (Korean language, 
complex sentences, clause-level syntax, discourse semantics, corpora (linguistics), interclausal relationships, connectives (linguistics))
}

\rule{0pt}{0pt}

\vfill

\emph{2008-08-08 01:02:38 +0900}
\end{document}




%%% Local Variables:
%%% End:
%%% TeX-header-end: "^% End-Of-Header$"
%%% TeX-trailer-start: "^% Start-Of-Trailer$"
%%% End:
