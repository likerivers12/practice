%% projectmain = mythesis.tex
%%% 국문초록
\begin{summary}
%
% 배경: 왜 시작했는가?
%

\par

한국어 복합문은 문어 텍스트에서 문장의 대다수를 차지하며 구어 발화에서도
상대적으로 비중은 낮으나 여전히 높은 빈도로 자연스럽게 사용된다. 이러한 복합문의
사용은 한국어의 특징적인 현상이다. 이론적인 차원에서 복합문의 통사와 의미는
국문법의 중요한 주제의 하나로 연구되어 왔으며 실제적인 차원에서 복합문의 구문
분석은 가장 어려운 문제의 하나로 인식되고 있다.

%
% 문제 제기
%

이러한 중요성에도 불구하고 다중절 복합문의 구조적 특성과 중의성 해소에
대해서는 깊은 언어학적 논의가 이루어지지 않았다. 기존의 연구들은 가장 단순한
구성인 이중절 복합문을 논의 대상으로 하며 문장 성분들 사이의 통사적 제약에
분석의 초점을 두고 있다. 세 개 이상의 절로 이루어진 복합문의 구조는 주요하게
다루어지지 않았다.


%
% 목적: 무엇을 하려고 했는가?
%

이 논문은 다중절 복합문을 대상으로 절과 절 사이의 관계를 분석하여 절 단위 계층
구조를 분석하는 것을 목표로 하였다. 이 논문은 다중 접속절 복합문의 구조를
이해하는 데에 관여하는 언어적 정보가 무엇인지 밝히는 것에 관심을 가지고 있다.
한국어 문법에서 한 절의 문법적 특성은 용언에 결합한 어미의 특성으로 파악되므로
어미가 복합문의 구조를 제약하는 현상을 분석하는 것이 제일의 목표가 되었다.
어미를 절을 이끄는 가장 대표적인 요소로 주요하게 다루되 접속어미뿐만 아니라
접속의 기능을 하는 `-을 때', `-는 동안', `-기 때문에' 등의 접속기능구성도 분석
대상으로 삼았다.

%
% 방법: 무엇을 했는가?
%

실제 분석에서는 21세기 세종계획의 현대국어 기초 말뭉치를 이용하였으며 다중
접속절 복합문의 통사 구조가 접속어미에 의해 제약되는 현상에 초점을 두고
문형 분석, 통사 분석, 의미 분석을 수행하였다. 이러한 분석은 언어 형태만을
고려하는 통계적인 분석에서 점차 상세한 문법 정보를 이용한 정밀한 관찰로
나아가며 수행된 것이다. 4장의 문형 분석에서는 복합문의 문형을 각 하위절을
이끄는 어미들의 연쇄 패턴으로 파악하여 분석하였고, 5장의 통사 분석에서는 삼중절
복합문의 통사 구조와 두 접속어미 사이의 위계 관계 분석을 기초로 하여 다중절
복합문에서 절 사이의 위계 관계를 분석하였으며, 6장의 의미 분석에서는 접속어미의
다의성에 따른 통사 구조의 제약을 분석하였다.

%
% 결과: 무엇을 발견했는가?
%

문형 분석 결과는 다중절 복합문에 어미들의 고정된 연쇄 패턴이 존재하지
않으며, 접속어미들의 선형 결합 양상이 매우 자유롭다는 것을 보여주었다.


통사 분석 결과는 접속어미 사이에 일정한 위계 관계가 존재하며 다중 접속절
복합문의 절 단위 계층 구조가 각 접속절을 이끄는 접속어미 사이의 위계 관계에
따라 결정됨을 보여주었다. 예를 들어 세 개의 절로 이루어진 문장 `S$_1$-$c_1$
S$_2$-$c_2$ S$_3$'의 구조는 접속어미 $c_1$와 $c_2$의 위계 관계에 따라
제약된다.

\ex.
\a. $c_1 < c_2$ $\rightarrow$ [[S$_1$-$c_1$ S$_2$-$c_2$] S$_3$]
\b. $c_1 > c_2$ $\rightarrow$ [S$_1$-$c_1$ [S$_2$-$c_2$ S$_3$]]


의미 분석 결과는 이러한 복합문의 구조적 제약 현상이 접속어미라는 문법 형태
사이의 위계 관계가 아니라 접속의 의미 사이의 위계 관계에 의한 것임을
보여주었다. 예를 들어 다의적 접속어미의 하나인 `-아서'가 인과 관계로 기능할
때와 계기 관계로 기능할 때 보이는 통사 구조의 차이가 이를 분명히 드러낸다.
 
이러한 분석 결과가 의미하는 것은 접속의 의미가 이루는 위계 구조가 절 단위 계층
구조를 제약한다는 것, 즉 절 수준의 의미가 복합문의 통사 구조를 제약한다는 것이다.
이 논문에서 관찰한 언어 자료의 테두리 내에서 접속의 의미들
사이에 다음과 같은 위계가 성립하는 것을 확인하였다.

\ex. 대조 > 양보 > 조건 > 중단 > 동시 > 계기

접속의 의미 사이의 위계는 담화 구조의 논리적 분석에서 사용하는 보편적인 개념을
빌어 일반화될 수 있다. 대조는 닮음 관계, 양보와 조건은 인과 관계, 중단, 동시,
계기는 인접 관계에 해당한다.


\ex. 닮음 관계 > 인과 관계 > 인접 관계


%
% 결론: 그게 무슨 의미인가?
%


복합문 내의 접속이 담화 내의 연결과 동일한 성격을 가진다는 사실은 다중절
복합문과 다중문 담화를 동일한 원리로 분석하는 방법론의 타당성을 뒷받침한다.
이를 바탕으로 이 논문에서는 접속어미를 함수로 보고 이에 의해 연결되는 두
절을 논항으로 보는 D-LTAG, SDRT, RST 등의 관점에 동의하며 한국어 복합문의 통사
구조와 의미 구조를 분석하고 형식화하기 위한 틀을 제안하였다.

\vfill

\bigskip
\noindent
주요어: 한국어, 복합문, 절 수준 통사, 담화 의미론, 말뭉치, 접속어미\\
학\hspace{11pt}번: 2003--30026

\end{summary}



%%% Local Variables: 
%%% mode: latex
%%% TeX-master: "mythesis"
%%% End: 
