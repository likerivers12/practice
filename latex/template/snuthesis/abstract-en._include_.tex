%% projectmain = mythesis.tex
\pagestyle{plain}
\setlength\parindent{1.5pc}% 들여쓰기 1글자
\begin{abstract}


\begin{center}
\normalsize\sc
{\maketitlebreakwork\bfseries
The Syntax and Semantics of Korean Complex Sentences:\\
\small
a corpus based study of the hierarchical structure of interclausal relations\\
\maketitlebreaknotwork
}

\bigskip\small
by\\
\normalsize
YOU Hyun Jo\\
\bigskip
\small
Under the Supervision of Professor KWON Jae-il\\
Submitted to the Department of Linguistics, Seoul National University,\\
in partial fulfillment of the requirements for the degree of\\
Doctor of Philosophy
\bigskip

August 2008
\end{center}
\bigskip

% Movitation
\noindent
Korean complex sentences are exceedingly dominant in written texts and
reportedly not dominant but still  frequent and natural in spoken
utterances. The verbal endings play the key role in the syntax and semantics
of the multi-clausal structures. The clause combining with `connective' verbal
endings or `converb' forms has been studied as one of the most important
syntactic phenomena since the existence of the traditional Korean grammar and
considered as one of the most challenging problems of parsing in recent
computational linguistic research.

% Problem statement
The syntax of multi-clausal sentences has hardly been studied in traditional
and modern grammars, in which the discussion has usually been limited to
describing two-clause sentences and the semantic function of verbal endings. A
complex sentence with one subordinate clause has the evident clause level syntax
but one with multiple subordinate clauses has enormous ambiguity exactly like
in the classical PP attachment problem.

%  What did you try to do? What did yo do?
In this paper I have attempted to uncover what linguistic information is
relevant to dealing with structural ambiguity of the complex sentences with
multiple adjunct `connective' clauses. In Korean grammar, the grammatical
properties of a clause are viewed as properties of the verb and its ending.
The corpus analysis was designed to test the hypothesis that the connectives
determine the clause level syntax.

This study investigated the restriction of the connective verbal endings on
the structure of the multi-clausal sentences in large corpora. I computed the
mutual information between the connective endings of adjacent clauses for
identifying the pattern of clause combining in chapter 4, and extracted
syntactic dependencies between connective clauses from parsed corpora for
constructing the hierarchy of connective endings in chapter 5, and examined
carefully how the polysemy of connective endings affect the syntactic
structure of the sentence using manually annotated sentences in chpter 6.

% What did you find
I found no clause combining patterns but consistent hierarchical relationships
between connective verbal endings. The relative ranks of the endings restrict
the clause combining. The syntactic structure of a sentence with two
subordinate clauses `S$_1$-$c_1$ S$_2$-$c_2$ S$_3$' is restricted by the
relationship between the two connectives $c_1$ and $c_2$: `[[S$_1$-$c_1$
S$_2$-$c_2$] S$_3$]' if $c_1 < c_2$, `[S$_1$-$c_1$ [S$_2$-$c_2$ S$_3$]]' if
$c_1 > c_2$. The hierarchy is determined according to the order of the
semantic functions of the connectives: Contrast > Concession > Condition >
Interruption > Concurrence > Succession. It can be generalized to the
hierarchy of coherence relations in the terms of David Hume: Resemblance >
Cause and Effect > Contiguity.

% What does it means?
It means that we can apply same rule to the analysis of the multi-clausal
sentential structures and the multi-sentential discourse structures. The rule
is a specification of the semantic requirement of the connective
(the predicate) on its argument (the clauses). I proposed some methods for dealing
with syntactic and semantic analysis of Korean complex sentences based on the
philosophy of D-LTAG (Discourse-Lexicalized Tree Adjoining Grammar), SDRT
(Segmented Discourse Representation Theory) and RST (Rhetorical Structure
Theory).

\vfill
\noindent
Keywords: Korean language, complex sentences, 
clause-level syntax, discourse semantics, corpora (linguistics), interclausal
relationships, connectives (linguistics)\\
Student Number: 2003--30026


\end{abstract}

%%% Local Variables: 
%%% mode: latex
%%% TeX-master: "mythesis"
%%% End: 
