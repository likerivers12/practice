%%%%%%%%%%%%%%%%%%%%%%%%%%%%%%%%%%%%%%%%%%%%%%%%%%%%%%%%
%%% Ch 2 텍스트의 조판

%\documentclass[korean,hangul,a4paper, 11pt]{article}
%\documentclass[a4paper, 11pt]{article}
\documentclass[a4paper, 11pt]{article}

% \usepackage{ucs}
% \usepackage[utf8]{inputenc}

% 한국어 문서
% p30
\usepackage[hangul, nonfrench]{dhucs}


\begin{document}



\section{}
This is a section
\\

%%%%%%%%%%%%%%%%%%%%%%%%%%%%%%%%
%%% ch 2-8 상호 참조
%%% p37

% label{marker}, \ref{marker} and \pageref{marker}

A reference to this subsection 
\label{sec:this} looks like:
``see section~\ref{sec:this} on
page~\pageref{sec:this}.''




%%%%%%%%%%%%%%%%%%%%%%%%%%%%%%%%
%%% ch 2-9 각주
%%% p38
Foot\footnote{This is a footnote.}
are often used by people using \LaTeX
\\


%%%%%%%%%%%%%%%%%%%%%%%%%%%%%%%%
%%% ch 2-10 강조
%%% p38

사랑했던 \underline{너를 잊지 못해}. 부디 너를 다시 볼 수 있다면 기다릴 수 있어. 잠시 멀리있는 거야.\\
\emph{안녕 슬픈 우리 사랑}. 삶이 끝나는 날까지 남아서 눈물이 된다는 마지막 고백 지울 수 없겠지.
\\

% p38
\emph{If you use emphasizing inside a piece of emphasized text, then \LaTeX{} uses the
\emph{normal} font for emphasizing.}



%%% p39
\textit{You can also 
  \emph{emphasize} text if it is set in italics,}
\textsf{in a
  \emph{sans-serif} font,}
\texttt{or in
  \emph{typewriter} style.}






\end{document}
