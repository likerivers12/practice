%%%%%%%%%%%%%%%%%%%%%%%%%%%%%%%%%%%%%%%%%%%%%%%%%%%%%%%%
%%% Ch 2 텍스트의 조판

%\documentclass[korean,hangul,a4paper, 11pt]{article}
%\documentclass[a4paper, 11pt]{article}
\documentclass[a4paper, 11pt]{article}

% \usepackage{ucs}
% \usepackage[utf8]{inputenc}

% 한국어 문서
% p30
\usepackage[hangul, nonfrench]{dhucs}


\begin{document}


%%%%%%%%%%%%%%%%%%%%%%%%%%%%%%%%
%%% ch 2-11 환경
%%% p39

% \begin{env} text \end{env}

%%%---------------------
%%% ch 2-11.1 Itemize, Enumerate, Description 환경
\flushleft
\begin{enumerate}
  \item You can mix the list environments to your taste:

  \begin{itemize}
    \item But it might start to look silly
    \item[-]With a dash.
  \end{itemize}
  \item Therefore remember:

  \begin{description}
    \item[Stupid] things, will not become smart because they are in a list.
    \item[Smart] things, though, can be presented beautifully in a list.
  \end{description}
\end{enumerate}


%%%---------------------
%%% ch 2-11.2 Flushleft, Flushright, Center 환경
\begin{flushleft}
  This text is\\ left-aligned. \LaTeX{} is not trying to make each line the same length.
\end{flushleft}

\begin{flushright}
  This text is right-\\aligned. \LaTeX{} is not trying to make each line the same length.
\end{flushright}

\begin{center}
  At the centre\\of the earth
\end{center}


%%%---------------------
%%% 2-11.3 Quote, Quotation, Verse 환경
%%% p40
A typographical rule of thumb for the line length is:
\begin{quote}
  On average, no line should be longer than 66 characters.
\end{quote}
This is why \LaTeX{} pages have such large borders by default and
also why multicolumn print is used in newspapers.
\\

I know only one English poem by heart. It is about Humpty Dumpty.
\begin{flushleft}
  \begin{verse}
    Humpty Dumpty sat on a wall:\\
    Humpty Dumpty had a great fall.\\
    All the King's horses and all the King's men\\
    Couldn't put Humpty together again.
  \end{verse}
\end{flushleft}


%%%---------------------
%%% 2-11.4 Abstract 환경
%%% p41
\begin{abstract}
  The abstract abstract.
\end{abstract}


%%%---------------------
%%% 2-11.5 Verbatim 환경
%%% p42

%%% \begin{verbatim} ~ \end{verbatim}
%%% 문서의 내용 그대로 출력

%%% 개별 단락안에서
%%% \verb+text+

The \verb|\ldots| command \ldots

\begin{verbatim}
10 PRINT "HELLO WORLD ";
20 GOTO 10
\end{verbatim}

\begin{verbatim*}
the starred version of the       verbatim
environment emphasizes the spaces    in the text
\end{verbatim*}

\verb*|like   this :-) |


%%%---------------------
%%% 2-11.6 Tabular 환경

%\begin{tabular}[pos]{table spec}

\begin{tabular}{|r|l|}
\hline
  7C0 & hexadecimal \\
  3700 & octal \\ 

  \cline{2-2}
  11111000000 & binary \\

\hline 
\hline
  1984 & decimal \\
\hline
\end{tabular}






\end{document}

