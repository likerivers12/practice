%%%%%%%%%%%%%%%%%%%%%%%%%%%%%%%%%%%%%%%%%%%%%%%%%%%%%%%%
%%% Ch 2 텍스트의 조판
%%% p17


%\documentclass[korean,hangul,a4paper, 11pt]{article}
\documentclass[a4paper, 11pt]{article}

\begin{document}


%%%%%%%%%%%%%%%%%%%%%%%%%%%%%%%%%%%%%%%%%%%%%%%%%%
%%% Ch 2-1 텍스트와 언어의 구조
%%% p17


% Example 1
\ldots when Einstein introduced his formula
\begin{equation}
  e = m \cdot c^2 \; ,
\end{equation}
which is at the same time the most widley known
and the least whell understood physical formula.



% Example 2
\ldots from which follows Kirchhoff's current law:
\begin{equation}
  \sum_{k=1}^{n} I_k = 0 \; .
\end{equation}
Kirchhoff's voltage law can be derived \ldots

% % Example 3
\ldots which has several advantages.

\begin{equation}
 I_D = I_F - I_R
\end{equation}
is the core of a very different transistor model. \ldots



%%%%%%%%%%%%%%%%%%%%%%%%%%%%%%%%%%%%%%%%%%%%%%%%%%
%%% Ch 2-2 줄바꿈과 쪽 나눔
%%% p19

%%% 2-2.1 단락 정렬
%%% p19

aaa
\\
bbb
\newline
ccc
\\*
ddd
\newpage
eee
% \linebreak[n],
% \nolinebreak[n],
% \pagebreak[n],
% \nopagebreak[n],





%%% 2-2.2 분철
%%% p20
aaaaa bbbbbb cccccccc ddddddddd eeeeeeeeeee fffffffff ggggggggggg hhhhhhhhh 
iiiiiiii jjjjjjjj kkkkkkkkkk lllllllll mmmmmmmmm nnnnnnnnn oooooooooo\
\\

\sloppy
aaaaa bbbbbb cccccccc ddddddddd eeeeeeeeeee fffffffff ggggggggggg hhhhhhhhh 
iiiiiiii jjjjjjjj kkkkkkkkkk lllllllll mmmmmmmmm nnnnnnnnn oooooooooo
\\

\fussy
aaaaa bbbbbb cccccccc ddddddddd eeeeeeeeeee fffffffff ggggggggggg hhhhhhhhh 
iiiiiiii jjjjjjjj kkkkkkkkkk lllllllll mmmmmmmmm nnnnnnnnn oooooooooo
\\

\hyphenation{FORTRAN Hy-phen-a-tion}

I think this is: su\-per\-cal\-%i\-frag\-i\-lis\-tic\-ex\-pi\-%
al\-i\-do\-cious


% 여러 낱말들을 줄바꿈 없이 같은 줄에 두고 싶을 때
% \mbox{text}
My phone number will change soon.
It will be \mbox{0116 291 2319}.

The parameter 
\mbox{\emph{filename}} should
contain the name of the file


%%%%%%%%%%%%%%%%%%%%%%%%%%%%%%%%%%%%%%%%%%%%%%%%%%
%%% Ch 2-3 미리 정의된 문자열
%%% p21

\today
\\

\TeX
\\

\LaTeX
\\

\LaTeXe
\\


%%%%%%%%%%%%%%%%%%%%%%%%%%%%%%%%%%%%%%%%%%%%%%%%%%
%%% Ch 2-4 특수 문자와 기호
%%% p21

%%% 2-4.1 따옴표
%%% p21

''Please press the 'x' key.''



%%% 2-4.2 대시와 하이픈
%%% p22

daughter-in-law, X-rated\\
pages 13-67\\
yes---or no?\\
$0$, $1$ and $-1$
\newline


%%% 2-4.3 틸데(~)
%%% p22

http://www.rich.edu/\~{}bush \\
http://www.clever.edu/$\sim$demo
\newline

% \url{http://people.ktug.or.kr/~karnes/
% demo/lshort-kr/index.html}


%%% 2-4.4 도 기호
%%% p23

It's $-30\,^{\circ}\mathrm{C}$.
I will soon start to super-conduct
\newline


%%% 2-4.5 유로 통화 기호
%%% p23,4

% \usepackage{textcomp}
% \texteuro


%%% 2-4.6 줄임표(...)
%%% p24

Note like this ... but like this: \\
New York, Tokyo, Budapest, \ldots
\newline



%%% 2-4.7 합자
%%% p25

\Large Not shelfful\\
but shelf\mbox{}ful
\newline


%%% 2-4.8 강세 부호와 특수 문자
%%% p25

H\^otel, na\"\i ve, \'el\`eve,\\
sm\o rrebr\o d, !`Se\~norita!,\\
Sch\"onbrunner Schol\ss{}
Stra\ss e
\newline


\`o \'o \^o \~o 
\=o \.o \"o \c c\\
\u o \v o \H o \c o
\d o \b o \t oo
\oe \OE \ae \AE\\
\aa \AA \o \O \l \L \i \j !` ?`



\end{document}
