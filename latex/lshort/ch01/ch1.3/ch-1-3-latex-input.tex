\documentclass{article}

% 패키지는 \documentclass ~ \begin 사이에.
\usepackage{verbatim}

\begin{document}

%%%%%%%%%%%%%%%%%%%%%%%%%%%%%%%%%%%%%%%%%%%%%%%%%%%
% ch3.3 LaTeX명령
% p6

I read that Knuth divides the people working with 
\TeX{} into \TeX{}nicians and \TeX perts. \\
Today is \today

You can \textsl{lean} on me!

Please, start
a new line right here!\newline
Thank you!


%%%%%%%%%%%%%%%%%%%%%%%%%%%%%%%%%%%%%%%%%%%%%%%%%%%
% ch3.4 주석
%p7

% \begin,end{comment}를 사용하려면 verbatim 패키지를 사용해야한다.
% \usepackage{verbatim}
%
This is another 
\begin{comment}
rather stupid,
but helpful
\end{comment}
example for embedding
comments in your document


%%%%%%%%%%%%%%%%%%%%%%%%%%%%%%%%%%%%%%%%%%%%%%%%%%%
% 제 4절 입력파일의 구조
% p7
%

% 시작       : \documentclass{...}
% 패키지 포함 : \usepackage{...}
% 끝        : \end{document}

Small is beautiful


%%%%%%%%%%%%%%%%%%%%%%%%%%%%%%%%%%%%%%%%%%%%%%%%%%%
% 제 5절 명령행 작업
% p8

command line - 

latex foo.tex

xdvi foo.dvi \&

dvips -Pcmz foo.dvi -o foo.ps

dvipdf foo.dvi


%%%%%%%%%%%%%%%%%%%%%%%%%%%%%%%%%%%%%%%%%%%%%%%%%%%
%%%%%%%%%%%%%%%%%%%%%%%%%%%%%%%%%%%%%%%%%%%%%%%%%%%




\end{document}
